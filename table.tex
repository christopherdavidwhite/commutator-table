\documentclass[12pt,letterpaper]{article}
\usepackage[utf8]{inputenc}
\usepackage[left=1in,right=1in,top=1in,bottom=1in]{geometry}
\usepackage{datatool}
\author{Nathaniel Beaver}
\title{A Table of Commutation Relations}

\usepackage[numbers]{natbib}
\usepackage[pdftex,bookmarks=true]{hyperref}
\hypersetup{
    pdftitle={A Table of Commutation Relations},
    pdfauthor={Nathaniel Beaver}
}

\usepackage{braket} %Bra ket notation

\begin{document}
\maketitle

\section{A table of commutation relations for operators of a single particle.}

\subsection{Operators to use.}

\begin{itemize}
\item Observables (Hermitian): $x, p_x, H=\frac{p_x^2}{2m}, N=a^\dagger a, J_x, J_y, J_z, J, \hyperref[def_Pi]{\Pi}$
\item Non-observables: $a, a^\dagger, T, $
\end{itemize}

\subsection{The table.}

%\begin{tabular}{ l | r r }
%            & $x$ & $p_x$ \\
%\hline
%  $x$ & \hyperref[self]{$0$} & \hyperref[unc]{$i \hbar$}\\
%  \hyperref[def_p_x]{$p_x$} & $-i\hbar$ & $0$ \\
%\end{tabular}
\DTLloaddb[keys={A,B,C,D,E,F,G,H,I,J,K,L,M}]{commutators}{commutation-relations.csv}
\renewcommand{\dtldisplayafterhead}{\hline} %
\renewcommand{\dtlstringalign}{c} % make strings default to center-aligned

%\renewcommand{\dtlbetweencols}{|} % this is fairly cosmetic, but I like it off
\DTLdisplaydb{commutators}

\DTLloaddb[keys={A,B,C,D}]{spin-one-half}{spin-one-half-system.csv}
\DTLdisplaydb{spin-one-half}

\newpage

\subsection{Operator definitions.}
\begin{itemize}
\item \label{def_p_x} $p_x = \frac{\hbar}{i}\frac{\partial}{\partial x}$ in the x-basis. Townsend p.158.
\item \label{def_Pi} $\Pi\ket{x}=\ket{-x}$ Townsend p.213.
\item \label{def_H} $H = \frac{p^2}{2m} + V(x)$ Sakurai p.97, (2.4.2). 
\end{itemize}

\subsection{References.}
\begin{itemize}
\item[0] \label{self} Any operator commutes with itself.
\item[$-i \hbar$] \label{unc} See Townsend eq. 6.31, p. 155 or Griffiths eq. 2.51, p.55. This leads to the uncertainty principle $\Delta x \Delta p_x \geq \frac{\hbar}{2}$.
\item[$-a$] \label{N-a} See Sakurai p.90, (2.3.10).
\item[$-a^\dagger$] \label{N-aDag} See Sakurai p.90, (2.3.11).
\item[0] \label{H-p_x} By the product rule.
\end{itemize}

\section{A table of commutation relations for operators of two particles.}

\end{document}
